\documentclass[a4paper]{article}
\usepackage[x11names]{xcolor}
\usepackage{tikz}
\usetikzlibrary{arrows,shadows}
\usepackage{pgf-umlsd}
\usepackage{fixltx2e}

% links
\usepackage{hyperref}
\hypersetup{%
  colorlinks=true,
  linkcolor=DodgerBlue1,
  urlcolor=DodgerBlue1,
  pdfauthor={$author-meta$},
  pdftitle={$title-meta$}
}

% http://tex.stackexchange.com/questions/46953/unix-command-highlighting-latex
\usepackage{listings}

\lstdefinestyle{BashInputStyle}{
  language=bash,
  basicstyle=\sffamily,
  frame=tb,
  columns=fullflexible,
  backgroundcolor=\color{yellow!5},
  linewidth=0.9\linewidth,
  xleftmargin=0.1\linewidth,
  morekeywords={checkout,clone,fetch,flow,merge,pull,push,rebase,remote}
}

\usepackage[top=1cm, bottom=1cm, left=2cm, right=2cm]{geometry}

% http://www.apreche.net/github-suggested-workflow/

\begin{document}
\renewcommand{\thesection}{\alph{section}.}

% A footnote-like superscript (comments at end of lines to trim whitespace)
\newcounter{MyCount}
\newcommand\step{%
	\stepcounter{MyCount}%
	\textsuperscript{\arabic{MyCount}}%
}

\begin{sequencediagram}

\tikzstyle{inststyle}+=[ top color = LightBlue1, bottom color = LightBlue2, font = \footnotesize ] % custom the style
\newthread[white]{p}{AuScope/AuScope-Portal}
\newinst[2]{f}{\textit{USER}/AuScope-Portal}
\tikzstyle{inststyle}+=[ top color = Honeydew1, bottom color = Honeydew2] % custom the style
\newinst[2]{c}{Clone}
\newinst[2]{b}{\textit{working}}

\begin{call}{p}{github::fork\step}{f}{github::pull\step}

  \begin{call}{f}{clone\step}{c}{push\step}

	\mess{p}{upstream\step}{c}

  	\begin{call}{c}{branch\step}{b}{merge\step}
	  	    \begin{callself}{b}{Do stuff}{commit}
	  	    \end{callself}

	  	    \mess{p}{pull\step}{c}
	  	    \mess{c}{rebase\step}{b}
  	\end{call}

  \end{call}
\end{call}

\end{sequencediagram}

\section{Fork in Github}

See Github help for \href{https://help.github.com/articles/fork-a-repo}{Fork A Repo}.

\section{Clone and initialise your fork}

% Prevent formatting of minus signs into math symbols
\lstset{language=bash,literate={-}{{-}}1}

\begin{lstlisting}[style=BashInputStyle]
	# (2) Clone the project fork
	git clone git@github.com:USER/AuScope-Portal.git
	cd AuScope-Portal
	# (3) Add the original project as the ``upstream'' source for changes and updates.
	git remote add upstream git@github.com:AuScope/AuScope-Portal.git
\end{lstlisting}


\section{Create a branch for your work}

\begin{lstlisting}[style=BashInputStyle]
	# (4) Create a branch for your work.
	git checkout -b working
\end{lstlisting}

\begin{center}
\fbox{\parbox{0.65\textwidth}{\emph{Make whatever changes you want, committing them to the ``working'' branch. When you are finished, merge upstream $\ldots$}}}
\end{center}

\section{Update and merge your code into the upstream project}

\begin{lstlisting}[style=BashInputStyle]
	# (5) Pull the latest Portal updates into the master branch of your project fork
	git checkout master
	git pull upstream master

	# (6) Rebase these changes into your working copy
	git checkout working
	git rebase master

	# (7) Merge the working updates and your changes into the master branch and
	# (8) Send them to Github
	git checkout master
	git merge working
	git push
	# optional (delete working branch): git branch -d working
\end{lstlisting}

\section{Create Github pull request}

See Github help for \href{https://help.github.com/articles/creating-a-pull-request}{Creating a pull request}.

\section*{Keeping up to date}

To keep your project fork up to date with the latest changes from upstream \texttt{AuScope/AuScope-Portal} do the following:

\begin{lstlisting}[style=BashInputStyle]
	git fetch upstream
	git checkout master
	git merge upstream/master
	git push origin master
\end{lstlisting}


\end{document}
